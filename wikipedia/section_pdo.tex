== El Festival de las Linternas ==

En 2024, se ejecuta en la ubicación un evento privado que ocupa más del 50% de las dimensiones del Parque, privando a la vecindad del uso y disfrute habitual, perdiendo acceso a baños públicos y la salida de emergencia... durante al menos cinco meses.

El montaje de este evento arrancaba a finales de octubre de 2024. El evento tenía su apertura 30 de noviembre de 2024<ref>[https://cadenaser.com/andalucia/2024/11/30/tensa-apertura-en-malaga-de-un-festival-de-luces-que-obliga-a-cerrar-al-publico-medio-parque-ser-malaga/ Tensa apertura en Málaga de un festival de luces que obliga a cerrar al público medio parque. ] Artículo en Cadena SER.</ref> y terminaba el 15 de febrero de 2025.<ref>[https://www.malagahoy.es/malaga/ultimo-dia-festival-linternas-malaga_0_2003343321.html Último día del Festival de las Linternas de Málaga entre protestas: "Parque sí, negocio no".] Artículo en MalagaHoy.</ref> El desmontaje del evento finalizaría el 15 de marzo de 2025.

Este llamado Festival es un acercamiento a la [[Fiesta de los Faroles]], con motivo de celebrar el [[Año Nuevo chino]]. 

Esto provoca una reacción en las vecindades en forma de un movimiento vecinal,<ref>[https://www.elsaltodiario.com/movimiento-vecinal/privatizacion-parque-oeste-malaga-vecinas-lucha Contra la privatización del Parque del Oeste de Málaga: vecinas en lucha] Artículo en el Salto Diario sobre el movimiento vecinal.</ref> llamado [[Movimiento vecinal por el Parque del Oeste]]. Este movimiento resultó orgánico y espontáneo, independiente de cualquier partido político, creado de la vecindad para la vecindad.

El mismo arquitecto y urbanista que trabajó en el parque, Eduardo Serrano, manifestó su preocupación por el uso que se hizo del Parque y el precedente que podía establecerse:

''Es un choque ya que ha partido en dos la estructura del parque que ahora ha perdido todo el sentido. Se han puesto unos tablones negros que suponen un gran impacto visual. Todo ello se ha realizado sin contar con los vecinos'' <ref>{{Cita web|url=https://cadenaser.com/andalucia/2024/12/09/eduardo-serrano-arquitecto-del-parque-del-oeste-el-festival-es-un-choque-y-me-preocupa-que-esta-privatizacion-se-extienda-por-malaga-ser-malaga/|título=Eduardo Serrano, arquitecto del Parque del Oeste: "El festival es un choque y me preocupa que esta privatización se extienda por Málaga"|fechaacceso=2025-02-21|apellido=Orellana|nombre=Jesús Sánchez|fecha=2024-12-09|sitioweb=Cadena SER|idioma=es-ES}}</ref>

El movimiento vecinal se organizó para presentarse todas las semanas, desde la apertura del evento, durante 13 semanas para manifestar, en un ambiente festivo, alegre y creativo, su disconformidad por el uso privatizador del espacio público.

En febrero de 2025, el Ayuntamiento de Málaga se comprometió por escrito a no repetir futuras ediciones de este evento en el Parque del Oeste.<ref>[https://www.laopiniondemalaga.es/malaga/2025/02/17/fracaso-festival-linternas-malaga-parque-oeste-114388540.html Resultado del Festival de las Linternas, La Opinión de Málaga] Fracaso del Festival de las Linternas en Málaga: no volverá al Parque del Oeste.</ref>