El Movimiento vecinal por el [[Parque del Oeste (Málaga)]] es una articulación vecinal surgida en el [[Distrito Carretera de Cádiz]] de [[Málaga]] en respuesta a la privatización de uno de los parques más importantes de la ciudad durante cinco meses.
[[Archivo:15 FEB 2025 Parque del Oeste (2).jpg|miniaturadeimagen|Manifestantes en concentración por liberar el Parque del Oeste]]

== Origen del movimiento vecinal ==

En octubre de 2024, el [[Ayuntamiento de Málaga]] cierra el paso de la zona central del Parque para ubicar el Festival de Las Linternas. El evento privado, promovido por las empresas [[Iluminaciones Ximénez]] y [[Zigong Lantern Group]], ocupa más del 50% de la superficie de la zona, rompiendo el sentido del tránsito y expulsando a la vecindad de la parte más comunitaria del Parque.

A partir de esto, la vecindad y las barriadas comienzan a organizarse de forma espontánea y orgánica para protestar y manifestar su desacuerdo con la privatización del parque público.

Se conforma el movimiento vecinal de una forma asamblearia, apartidista, llevada por la vecindad para el bienestar de la vecindad.

Con asambleas cada miércoles y concentraciones los domingos, la vecindad encontró la forma de hacerse escuchar hasta liberar el parque y que no se volviese a repetir tal evento en el mismo.

== Sobre el evento ==

El parque público fue ocupado por un evento llamado el Festival de las Linternas. Este llamado Festival es un acercamiento a la [[Fiesta de los Faroles]], con motivo de celebrar el [[Año Nuevo chino]]. 

Este evento se replica de una forma similar en otras localizaciones alrededor del mundo.

Una de las empresas organizadoras de este evento, [[Iluminaciones Ximénez]], es la encargada de la iluminación navideña en [[Calle Larios]] de [[Málaga]] o en otras ciudades como [[Bilbao]], estando ambas en [[España]].

== Motivos detrás del movimiento ==

Además del uso privatizador del espacio público que es el Parque del Oeste, existen muchos motivos por los cuales el movimiento vecinal se organizó y tomó tanta fuerza.

La lista de estos motivos es, pero no se limita a, a los siguientes elementos:

* '''El evento ocupaba más del 55% del espacio del parque''', limitando el acceso y dividiendo el espacio público esencial para el disfrute de la vecindad. Se perdió además el sentido del diseño del Parque
* '''El espacio se usa durante todo el año''' y durante todas las horas que abre el espacio. Para escoger esta localización, se argumentaba que el parque no se usaba en invierno.
* El evento privaría del uso del parque '''durante al menos cinco meses, sumando tiempo de montaje, duración del evento y desmontaje de la infraestructura'''. Según diversas publicaciones científicas, la [[Organización Mundial de la Salud]] y el Ayuntamiento de Málaga, la herramienta 3-30-300 por la cual se mide la calidad del medio ambiente urbano se centra en la proximidad de la ciudadanía a los parques y zonas verdes, el acceso a ellas y su uso recreativo, impacta positivamente en la salud física y mental de las personas.
* La forma en la que se accede, atraviesa y disfruta del parque se ve alterada. Se perdieron acceso a las zonas verdes, más amplias y espaciosas, también se perdió acceso a uno de los baños públicos del parque.
* '''Se perdió el acceso a la única salida de emergencia''' del Parque, teniendo que construir una nueva a marchas forzadas, con ciertas dudas sobre la accesibilidad de la misma.
* La zona, el distrito, el parque público... no necesitan dinamización, argumento que se usó a favor de establecer este evento en el Parque del Oeste. Un parque público no necesita ser dinamizado ni necesita ser explotado para el beneficio de empresas.
* '''Se pierden formas de viajar de forma segura''' entre centros educativos, hogares, negocios... Se fuerza a personas de todas las edades, estudiantes, personas en toda clase de circunstancias y diversidades... a rodear el Festival, en lugar de atravesar el Parque, que es una forma mucho más segura de viajar.

== Eventos de interés ==

El 10 de noviembre de 2024, se produce la primera concentración de la vecindad para manifestar su desacuerdo por la privatización del Parque del Oeste. <ref>[https://www.malagahoy.es/malaga/vecinos-concentran-pedir-no-cierre-parque-oeste-malaga-festival-linternas_0_2002760818.html Decenas de vecinos se concentran para pedir que no cierre el Parque del Oeste de Málaga] Artículo en Málaga Hoy</ref>

El 30 de noviembre de 2024 es el estreno del Festival de las Linternas, en el que se produce una respuesta por parte de la vecindad en forma de concentración. <ref>[https://cadenaser.com/andalucia/2024/11/30/tensa-apertura-en-malaga-de-un-festival-de-luces-que-obliga-a-cerrar-al-publico-medio-parque-ser-malaga/ Tensa apertura en Málaga de un festival de luces que obliga a cerrar al público medio parque] Artículo en Cadena SER</ref> Esta respuesta es producto de la privatización del parque, el nulo o escaso consenso de las asociaciones vecinales y la instalación de paneles negros (para evitar que las exposiciones del Festival fueran vistas desde el exterior) alrededor del evento. La concentración tuvo una fuerte presencial policial.

Las concentraciones y asambleas se prolongaron durante toda la duración del evento, y más allá. En algunas ocasiones incluso se dieron intentos de censurar el movimiento, prohibiendo el uso de megafonía en concentraciones<ref>[https://cadenaser.com/andalucia/2025/01/03/la-policia-prohibe-a-los-vecinos-usar-megafonos-en-las-concentraciones-contra-el-festival-de-las-linternas-de-malaga-ser-malaga/ La policía prohíbe a los vecinos usar megáfonos en las concentraciones contra el festival de las linternas de Málaga] Artículo en Cadena SER</ref> o disolviendo una asamblea vecinal y pacífica de la vecindad.

El día 15 de febrero de 2025, fue el último día en el que el Festival de las Linternas estaría abierto, ocasión en la que el movimiento daba su última concentración para celebrar el final de tal atropello al parque público.<ref>[https://www.eldiario.es/andalucia/malaga/vecinos-celebran-festival-linternas-evento-privatizo-parque-publico-malaga_1_12057484.html Los vecinos celebran el fin del Festival de las Linternas, el evento que privatizó un parque público de Málaga] Artículo en elDiario.es</ref>

En febrero de 2025, el Ayuntamiento de Málaga se comprometió a no repetir futuras ediciones de este evento en el Parque del Oeste,<ref>[https://www.areacostadelsol.com/2025/02/11/renuncian-a-celebrar-el-festival-de-las-linternas-de-malaga-en-2025-tras-la-presion-vecinal/ Renuncian a celebrar el Festival de las Linternas de Málaga en 2025 tras la presión vecinal] Artículo en Área Costa del Sol</ref> pero no liberaba a otros parques públicos de sufrir la presencia del Festival de las Linternas o cualquier otro evento privatizador de la misma naturaleza.

El 24 de febrero se realiza la entrega de firmas recogidas durante el movimiento en el [[Ayuntamiento de Málaga]], en la que se entregan 5.623 firmas de personas que solicitan que tal evento no se vuelva a realizar en el Parque del Oeste.<ref>[https://cadenaser.com/andalucia/2025/02/24/el-movimiento-vecinal-entrega-5623-firmas-para-decir-no-al-festival-de-las-linternas-en-el-parque-del-oeste-ser-malaga/ El movimiento vecinal entrega 5.623 firmas para decir "no" al Festival de las Linternas en el Parque del Oeste] Artículo en Cadena SER</ref>

== Referencias ==
{{listaref}}

{{Control de autoridades}}
[[Categoría:Movimientos sociales]]
[[Categoría:Málaga]]
